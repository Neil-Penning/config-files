\begin{newProblem}{Name of problem}
    \begin{newInlineProblem}[NOTE]
    Note that tcblower matches the color of the problem
    \tcblower
    Other text
    \end{newInlineProblem}
\tcblower

\end{newProblem}
\begin{newProblem}[UNFINISHED]{}
    \begin{newInlineProblem}{A}
    \tcblower
        \begin{newInlineProblem}{A(i)}
        \tcblower
        \end{newInlineProblem}
        \begin{newInlineProblem}{A(ii)}
        \tcblower
        \end{newInlineProblem}
        \begin{newInlineProblem}[OPTIONAL]{A(iii)}
        \tcblower
        \end{newInlineProblem}
    \end{newInlineProblem}
    \begin{newInlineProblem}[UNFINISHED]{B}
    \tcblower
    \end{newInlineProblem}
    \begin{newInlineProblem}[UNFINISHED]{C}
    \tcblower
    \end{newInlineProblem}
    \begin{newInlineProblem}[OPTIONAL]{D}
    \tcblower 
    \end{newInlineProblem}
\end{newProblem}

\begin{newProblem}
\begin{newInlineProblem}[UNFINISHED]
\tcblower 
\end{newInlineProblem}
\tcblower
\begin{newInlineProblem}[NOTE]
\tcblower
\end{newInlineProblem}
\end{newProblem}

% Matches color of last defined syntax, unsure why
\tcblower

\begin{boxDefinition}{}
    \tcblower
\end{boxDefinition}
\begin{boxTheorem}{}
    \tcblower
\end{boxTheorem}
\begin{boxExample}{}
    \tcblower
\end{boxExample}
\begin{boxConcept}{}
    \tcblower
\end{boxConcept}
\begin{boxProblem}{}
    \tcblower
\end{boxProblem}
\begin{boxTODO}{}
    \tcblower
\end{boxTODO}
\begin{boxQuote}{}
    \begin{boxDefinition}{}
    \end{boxDefinition}
    \tcblower
    \begin{boxProblem}{}
    \end{boxProblem}
\end{boxQuote}

